\chapter{INTRODUCTION}~\label{ch:intro}
\raggedright\parindent=25pt
Science as a core human pursuit, aims to understand the natural world through observation, experimentation and analysis. The Large Hadron Collider, with its 27-kilometer ring of magnets, enable us to accelerate particles to nearly the speed of light and collide them with energy to replicate the conditions of the early universe. It is now known as the largest experiment mankind has ever built to probe the fundamental constituents of matter and the forces governing their interactions. The scientific, computational and engineering challenges bring about benefits to society beyond the understanding of the cosmos through technical innovations from the world-wide web to novel medical devices. Most importantly, it serves as a model for large-scale multinational collaboration that transcends geopolitical boundaries.

% Science, as a core human pursuit, aims to understand the natural world through observation, experimentation, and analysis. The Large Hadron Collider (LHC)~\cite{Evans:2008zzb} epitomizes modern particle physics experiments with its advanced engineering and technology. The LHC's 27-kilometer ring of magnets and cutting-edge computational capabilities enable it to accelerate particles to nearly the speed of light and collide them with immense energy replicating the conditions in the beginning of the universe. Through these collisions, scientists probe the fundamental constituents of matter and the forces governing their interactions. Beyond understanding the cosmos, the LHC's technological innovations also impacts society.  

On July 4, 2012, two of LHC's independent experiments, ATLAS and CMS, announced the discovery of a new fundamental particle consistent with the Higgs Boson. This discovery, predicted by theorists half a century earlier, filled a critical gap in the Standard Model (SM) of Particle Physics. Despite the LHC's primary mission to discover the Higgs Boson, it continues to operate, supported by global investment in upgrades. With over 20,000 scientists and engineers worldwide, research at the LHC aims to explore new physics. While the Higgs discovery provided insights into particle mass acquisition within the Standard Model, fundamental issues persist. The SM hierarchy problem, highlighted by the disparity between the electroweak and Planck scales, remains unresolved. This discrepancy, along with the significant weakness of gravity compared to the electroweak force, motivates exploration of Beyond-the-Standard-Model (BSM) frameworks to address foundational questions and provide a quantum description of gravity.


% Science is one of many human endeavours which allows us to understand the natural world through observation, experiment and analysis. At the forefront is the Large Hadron Collider (LHC)~\cite{Evans:2008zzb} the largest experiment on earth u the most advanced engineering and cutting-edge computational techniques, to delve into the deepest mysteries of the cosmos whilst impacting society with its many innovations. Its 27-km ring-of-magnets, accelerates particles near the speed of light and collides them to probe the fundamental constituents of matter and the forces governing their interactions.  

% On July 4, 2012, two independent experiments, ATLAS and CMS, announced their discoveries of a new fundamental particle consistent with the Higgs Boson. It was long considered to be the final piece of the puzzle in the Standard Model (SM) of Particle Physics as its existence was predicted by theorists 50 years ago. It has never been found until the LHC was built. While the LHC was built primarily to search for the long-sought particle, the Higgs Boson the machine continues to operate and nations all over the world continue to invest money and manpower to upgrade the machine. Over 20,0000 scientists and engineers spanning hundreds of institutes worldwide, continue to search for New Physics. The Higgs discovery filled a major gap in our understanding of the universe through the Standard Model, such as how elementary particles acquire mass. However, there are still fundamental issues that remain unresolved. To explore these problems, the search space is extended via the extension of the energy frontier or maximizing insights obtained from data through improved computational techniques and algorithms. 

% It is important to note that the SM has been a remarkably successful framework explaining a plethora of experimental results with great precision. It is nevertheless incomplete. A leading motivation for New Physics or Beyond-the-SM (BSM) physics is the SM hierarchy problem~\cite{Weinberg:1975gm,Susskind:1978ms}, where quantum corrections to the Higgs boson mass must be fine-tuned to $~(M_{EW}/M_{Pl})^2$ to keep it at the observed electroweak (EWK) scale where $M_{Pl}$ and $M_{EW}$ are the Planck and electroweak scales, respectively. Another way this problem is described is obvious discrepancy between the strength of gravity with the electroweak strength, gravity being over 10 orders of magnitude weaker than Higgs and EWK. Over the last decades, physicists have contemplated different Beyond Standard Model theoretical frameworks to explain the origin of this large difference, known as the hierarchy problem, and additionally provide a quantum description of gravity.

It is in this light that we continue to explore ideas that fundamentally rethink the structure of the universe as we know it. These ideas include extra dimensions with various geometries, and a distinct but related scenario in the form of the existence of a yet-to-be-seen scale invariant sector where strange hazy objects called Unparticles live. In this dissertation, discuss the measurable physical consequences of these new theoretical frameworks. We search for deviations from the predictions of the Standard Model in high-mass $\gamma\gamma$ events that could be characterized by these models. Here we focus on reporting our search for broad, non-resonant signatures which are expected from large extra spatial dimensions, the Continuum Clockwork mechanism and the Unparticles model. Large extra spatial dimensions models proposed by Arkani-Hamed, Dimopoulos and Dvali (ADD)~\cite{ArkaniHamed:1998rs, Antoniadis:1998ig, ArkaniHamed:1998nn} modify the fundamental Planck scale to be the same order as the electroweak scale. In their models, the SM fields are constrained to the usual 3+1 spacetime dimensions while gravity can also propagate in the \nED additional, compactified spatial dimensions. With these, the effective strength of gravity is reduced by a Gauss's law reduction in the flux. Not included in this document is the complementary search for the Randall and Sundrum (RS)~\cite{Randall:1999ee,Randall:1999vf} model with a 1D-warped extra dimension, where the signal is a resonant bump. Both ADD and RS scenarios, Kaluza--Klein (KK) modes of the graviton couple to the SM through the stress-energy tensor and decay into two SM particles. In ADD, the KK modes are closely spaced and result in a continuum or broad excess of diphoton events over the expected SM background. In the RS graviton model, the KK modes are on-shell and present as resolvable resonances in the diphoton mass spectrum.

Another proposed solution to the SM hierarchy problem is the continuum clockwork mech\-anism~\cite{Giudice:2016yja}.
In the continuum limit of the clockwork, the massless graviton is accompanied by an infinite tower of massive spin-2 graviton KK modes with a characteristic pattern of masses and couplings.
Like in the ADD extra dimension scenario, an approximately continuous distribution of KK modes results in a continuum excess~\cite{5LittleStringTheoryAtATeV, Giudiceetal}; however, in the clockwork scenario, the KK modes are all on-shell, while in the ADD case, virtual contributions to the diphoton spectrum are significant~\cite{Giudice:2003tu,Giudice:2004mg,Franceschini:2011wr}.

Searches on these phenomena span over 10 years of work from various independent experiments. Previous searches for BSM physics in high-mass diphoton events from Run 2 of the CERN LHC were performed by the ATLAS and CMS experiments using proton-proton collisions at the same center-of-mass energy $\sqrt{s}=13\TeV$~\cite{Sirunyan:2018wnk,Aad:2021,Aaboud:2017yyg,Khachatryan:2016yec,Aaboud:2016tru,Khachatryan:2016hje}. Prior searches have also been performed by both experiments in Run 1 of the LHC~\cite{Aad:2015mna,Khachatryan:2015qba,Aad:2012cy,Chatrchyan:2011fq,Chatrchyan:2011jx}, at $\sqrt{s}=7$ and 8\TeV, as well at the Fermilab Tevatron by the CDF~\cite{Aaltonen:2011xp,CDF:2002hrr,CDF:2010muc,CDF:2011weq} and D0~\cite{Abazov:2010xh,D0:2000cve,D0:2008hxb,D0:2005srl} collaborations using proton-antiproton ($p\Bar{p}$) collisions at $\sqrt{s}=1.96\TeV$.

 This dissertation also includes notes on the author's on-site contributions to upgrading the CMS Hadron Calorimeter (HCAL) from 2020 to 2022. These include upgrades in the HCAL Online Software (HCOS) to support the creation of a trigger for Long-Lived Particles as well as miscellaneous hardware upgrades. This contribution represents just one small part of the broader initiative to enhance the detector's sensitivity and longevity. Beyond the scope of this dissertation are multiple other efforts to uncover new physics in more complex, unexplored final states, improving detector sensitivity, and ensuring uninterrupted data-taking.

The remainder of this dissertation is organized as follows. Chapter~\ref{ch:theory} introduces the theoretical basis of the Standard Model, its limitations, and the theoretical frameworks with non-resonant signatures - the ADD Large Extra Dimensions, Clockwork/Linear Dilaton and the Unparticles, that we seek to explore. Chapter~\ref{ch:CMSExperiment} gives a description of the experimental setup and the some of the upgrades implemented in the Large Hadron Collider. Chapters~\ref{ch:background} and ~\ref{ch:SignalModelling}, discuss the Monte Carlo Background and Signal Modelling, respectively. We discuss the systematic uncertainties involved in the experiment in Chapter~\ref{ch:systematics}. In Chapter~\ref{ch:results}, we discuss the statistical interpretation of our findings and conclude the document in Chapter~\ref{ch:conclusion}.
% \chapter{\label{ch:intro}\textnormal{Introduction}}

% Science is one of many human endeavours and the Large Hadron Collider (LHC)~\cite{Evans:2008zzb} represents the current pinnacle of particle physics research with its colossal scale. Using the most advanced engineering and cutting-edge computational techniques, the LHC delves into the deepest mysteries of the cosmos whilst impacting society with its many innovations. Its 27-km ring-of-magnets, accelerates particles near the speed of light and collides them to probe the fundamental constituents of matter and the forces governing their interactions.  

% On July 4, 2012, two independent experiments, ATLAS and CMS, announced their discoveries of a new fundamental particle consistent with the Higgs Boson. It was long considered to be the final piece of the puzzle in the Standard Model (SM) of Particle Physics as its existence was predicted by theorists 50 years ago. It has never been found until the LHC was built. While the LHC was built primarily to search for the long-sought particle, the Higgs Boson the machine continues to operate and nations all over the world continue to invest money and manpower to upgrade the machine. Over 20,0000 scientists and engineers spanning hundreds of institutes worldwide, continue to search for New Physics. The Higgs discovery filled a major gap in our understanding of the universe through the Standard Model, such as how elementary particles acquire mass. However, there are still fundamental issues that remain unresolved. To explore these problems, the search space is extended via the extension of the energy frontier or maximizing insights obtained from data through improved computational analysis. 

% It is important to note that the SM has been a remarkably successful framework explaining a plethora of experimental results with great precision. It is nevertheless incomplete. A leading motivation for New Physics or Beyond-the-SM (BSM) physics is the SM hierarchy problem~\cite{Weinberg:1975gm,Susskind:1978ms}, where quantum corrections to the Higgs boson mass must be fine-tuned to $~(M_{EW}/M_{Pl})^2$ to keep it at the observed electroweak (EWK) scale where $M_{Pl}$ and $M_{EW}$ are the Planck and electroweak scales, respectively. Another way this problem is described is obvious discrepancy between the strength of gravity with the electroweak strength, gravity being over 10 orders of magnitude weaker than Higgs and EWK. Over the last decades, physicists have contemplated different Beyond Standard Model theoretical frameworks to explain the origin of this large difference, known as the hierarchy problem, and additionally provide a quantum description of gravity.

% It is in this light that we explore ideas that fundamentally rethink the structure of the universe as we know it. These ideas include extra dimensions with various geometries, and a distinct but related scenario in the form of the existence of a yet-to-be-seen scale invariant sector where strange hazy objects called Unparticles live. In this dissertation, discuss the measurable physical consequences of these new theoretical frameworks. We search for deviations from the predictions of the Standard Model in high-mass $\gamma\gamma$ events that could be characterized by these models. Here we focus on reporting our search for broad, non-resonant signatures which are expected from large extra spatial dimensions, the Continuum Clockwork mechanism and the Unparticles model. Large extra spatial dimensions models proposed by Arkani-Hamed, Dimopoulos and Dvali (ADD)~\cite{ArkaniHamed:1998rs, ArkaniHamed:1998nn, AntoniadisFermi1998} modify the fundamental Planck scale to be the same order as the electroweak scale. In their models, the SM fields are constrained to the usual 3+1 spacetime dimensions while gravity can also propagate in the \nED additional, compactified spatial dimensions. With these, the effective strength of gravity is reduced by a Gauss's law reduction in the flux. Not included in this document is the complementary search for the Randall and Sundrum (RS)~\cite{Randall:1999ee,Randall:1999vf} model with a 1D-warped extra dimension, where the signal is a resonant bump. Both ADD and RS scenarios, Kaluza--Klein (KK) modes of the graviton couple to the SM through the stress-energy tensor and decay into two SM particles. In ADD, the KK modes are closely spaced and result in a continuum or broad excess of diphoton events over the expected SM background. In the RS graviton model, the KK modes are on-shell and present as resolvable resonances in the diphoton mass spectrum.

% Another proposed solution to the SM hierarchy problem is the continuum clockwork mech\-anism~\cite{Giudice:2016yja}.
% In the continuum limit of the clockwork, the massless graviton is accompanied by an infinite tower of massive spin-2 graviton KK modes with a characteristic pattern of masses and couplings.
% Like in the ADD extra dimension scenario, an approximately continuous distribution of KK modes results in a continuum excess~\cite{5LittleStringTheoryAtATeV, Giudiceetal}; however, in the clockwork scenario, the KK modes are all on-shell, while in the ADD case, virtual contributions to the diphoton spectrum are significant~\cite{Giudice:2003tu,Giudice:2004mg,Franceschini:2011wr}.

% Searches for BSM physics in high-mass diphoton events from Run 2 of the CERN LHC were performed by the ATLAS and CMS experiments using proton-proton ($\Pp\Pp$) collisions at a center-of-mass energy $\sqrt{s}=13\TeV$~\cite{Sirunyan:2018wnk,Aad:2021,Aaboud:2017yyg,Khachatryan:2016yec,Aaboud:2016tru,Khachatryan:2016hje}. Prior searches have also been performed by both experiments in Run 1 of the LHC~\cite{Aad:2015mna,Khachatryan:2015qba,Aad:2012cy,Chatrchyan:2011fq,Chatrchyan:2011jx}, at $\sqrt{s}=7$ and 8\TeV, and also at the Fermilab Tevatron by the CDF~\cite{Aaltonen:2011xp,CDF:2002hrr,CDF:2010muc,CDF:2011weq} and D0~\cite{Abazov:2010xh,D0:2000cve,D0:2008hxb,D0:2005srl} experiments using $\Pp\Pap$ collisions at $\sqrt{s}=1.96\TeV$.



%  This dissertation also includes notes on the author's on-site contributions to upgrading the CMS Hadron Calorimeter (HCAL) from 2020 to 2022. These include upgrades in the HCAL Online Software (HCOS) to support the creation of a trigger for Long-Lived Particles as well as miscellaneous hardware upgrades. This contribution represents just one small part of the broader initiative to enhance the detector's sensitivity and longevity. Beyond the scope of this dissertation are multiple other efforts to uncover new physics in more complex, unexplored final states, improving detector sensitivity, and ensuring uninterrupted data-taking.

% The remainder of this dissertation is organized as follows. Chapter~\ref{ch:theory} introduces the theoretical basis of the Standard Model, its limitations, and the theoretical frameworks with non-resonant signatures - the ADD Large Extra Dimensions, Clockwork/Linear Dilaton and the Unparticles, that we seek to explore. Chapter~\ref{ch:CMSExperiment} gives a description of the experimental setup and the some of the upgrades implemented in the Large Hadron Collider. Chapters~\ref{ch:background} and ~\ref{ch:SignalModelling}, discuss the Monte Carlo Background and Signal Modelling, respectively. We discuss the systematic uncertainties involved in the experiment in Chapter~\ref{ch:systematics}. In Chapter~\ref{ch:results}, we discuss the statistical interpretation of our findings and conclude the document in Chapter~\ref{ch:conclusion}.


% Large extra spatial dimensions have been proposed to resolve this problem by modifying the fundamental Planck scale.
% Should the modified scale be of the same order as the electroweak scale, then little fine-tuning would be needed.
% In the model proposed by Arkani-Hamed, Dimopoulos, and Dvali (ADD)~\cite{ArkaniHamed:1998rs,Antoniadis:1998ig,ArkaniHamed:1998nn}, the SM fields are constrained to the usual 3+1 spacetime dimensions, while gravity can also propagate in \nED additional, compactified spatial dimensions.
% The effective strength of gravity is thus reduced by a Gauss's law reduction in the flux.




% While the standard model (SM) of particle physics has been a remarkably successful framework, it is widely expected to be incomplete.
% A leading motivation for beyond-the-SM (BSM) physics has been the SM hierarchy problem~\cite{Weinberg:1975gm,Susskind:1978ms}, whereby quantum corrections to the Higgs boson mass must be fine-tuned to $\sim(\Mew/\Mpl)^2$ to keep it at the observed electroweak scale, where $\Mpl$ and $\Mew$ are the Planck and electroweak scales, respectively.

% Large extra spatial dimensions have been proposed to resolve this problem by modifying the fundamental Planck scale.
% Should the modified scale be of the same order as the electroweak scale, then little fine-tuning would be needed.
% In the model proposed by Arkani-Hamed, Dimopoulos, and Dvali (ADD)~\cite{ArkaniHamed:1998rs,Antoniadis:1998ig,ArkaniHamed:1998nn}, the SM fields are constrained to the usual 3+1 spacetime dimensions, while gravity can also propagate in \nED additional, compactified spatial dimensions.
% The effective strength of gravity is thus reduced by a Gauss's law reduction in the flux.

% In the model proposed by Randall and Sundrum (RS)~\cite{Randall:1999ee,Randall:1999vf}, there is just one additional dimension but with a warped geometry, described by a curvature parameter $k$.
% The effective strength of gravity is exponentially suppressed by the curvature over the distance in the extra dimension between the Planck ``brane'' where gravity originates and the SM brane, to which SM fields are constrained.
% In both the ADD and RS scenarios, Kaluza--Klein (KK) modes of the graviton couple to the SM through the stress-energy tensor and decay into two SM particles.
% In ADD extra dimensions, the KK modes are closely spaced and result in a continuum excess of diphoton events over the expected SM background.
% In the RS graviton model, the KK modes are on-shell and present as resolvable resonances in the diphoton mass spectrum.

% Another proposed solution to the SM hierarchy problem is the continuum clockwork mech\-anism~\cite{Giudice:2016yja}.
% In the continuum limit of the clockwork, the massless graviton is accompanied by an infinite tower of massive spin-2 graviton KK modes with a characteristic pattern of masses and couplings.
% Like in the ADD extra dimension scenario, an approximately continuous distribution of KK modes results in a continuum excess~\cite{5LittleStringTheoryAtATeV,Giudiceetal}; however, in the clockwork scenario, the KK modes are all on-shell, while in the ADD case, virtual contributions to the diphoton spectrum are significant~\cite{Giudice:2003tu,Giudice:2004mg,Franceschini:2011wr}.

% Finally, in addition to the above models that could address the hierarchy problem, high-mass diphoton events are also potentially sensitive to other BSM physics, such as the decays of heavy spin-0 resonances. These spin-0 resonances could arise from extended Higgs sectors~\cite{Branco:2011iw,PhysRevD.8.1226,Craig:2013hca}.

% Searches for BSM physics in high-mass diphoton events from Run 2 of the CERN LHC were performed by the ATLAS and CMS experiments using proton-proton ($\Pp\Pp$) collisions at a center-of-mass energy $\sqrt{s}=13\TeV$~\cite{Sirunyan:2018wnk,Aad:2021,Aaboud:2017yyg,Khachatryan:2016yec,Aaboud:2016tru,Khachatryan:2016hje}. Prior searches have also been performed by both experiments in Run 1 of the LHC~\cite{Aad:2015mna,Khachatryan:2015qba,Aad:2012cy,Chatrchyan:2011fq,Chatrchyan:2011jx}, at $\sqrt{s}=7$ and 8\TeV, and also at the Fermilab Tevatron by the CDF~\cite{Aaltonen:2011xp,CDF:2002hrr,CDF:2010muc,CDF:2011weq} and D0~\cite{Abazov:2010xh,D0:2000cve,D0:2008hxb,D0:2005srl} experiments using $\Pp\Pap$ collisions at $\sqrt{s}=1.96\TeV$.

% We present the results of a search for BSM physics in high-mass diphoton events from $\Pp\Pp$ collisions at $\sqrt{s}=13$~TeV, using data collected with the CMS detector during the LHC Run~2 period of 2016--2018, corresponding to an integrated luminosity of 138~\fbinv.
% To search for both resonant and nonresonant deviations from the SM, we make use of two complementary background estimation techniques.
% In the search for resonant excesses from both the RS and heavy Higgs boson models, we implement a technique where the diphoton spectrum is fit to a parameterized functional form, allowing for a description of the shape based exclusively on data.
% In the search for nonresonant excesses that arise from the ADD and clockwork models, we use a next-to-next-to-leading order (NNLO) calculation in quantum chromodynamics of the SM diphoton background, and we estimate the background from jets being misidentified as photons using control samples in data.



% he Large Hadron Collider (LHC) at CERN stands as one of humanity's most remarkable scientific endeavors, representing the pinnacle of particle physics research. With its colossal scale and unparalleled capabilities, the LHC delves into the deepest mysteries of the universe, probing the fundamental constituents of matter and the forces that govern their interactions. As the world's most powerful particle accelerator, it has provided unprecedented insights into the nature of reality, from elucidating the elusive Higgs boson to exploring exotic particles and phenomena beyond the Standard Model. This dissertation embarks on a journey through the intricate realm of particle physics, leveraging the vast wealth of data generated by the LHC to unravel the mysteries of the cosmos and advance our understanding of the fundamental laws that govern the universe.

% On July 4, 2012, the ATLAS and CMS experiments famously announced their independent discovery of a new fundamental particle, the Higgs Boson. This discovery led to the 2013 Nobel prize jointly awarded to Francois Englert and Peter Higgs, two of the proponents of the Higgs mechanism, which contributes to our understanding to the origin of mass of subatomic particles. It was long considered to be the final piece of the puzzle in the Standard Model of Particle Physics as its existence was predicted by theorists 50 years but had never been found until the LHC was built.


% While the LHC was built primarily to search for the long-sought particle, the machine continues to operate and its capabilities are further enhanced with upgrades. With over 20,000 scientists and engineers spanning hundreds of institutes worldwide and funded by multiple nations, they collectively pool their resources in pursuit of answers to the lingering questions that the Standard Model has yet to address. 

% In this dissertation we search for new physics in the high-mass diphoton channel from Run 2 of the LHC using proton-proton collisions at a center-of-mass energy $\sqrt{s} = 13$ TeV. We present improved limits results from prior searches performed by both ATLAS and CMS in Run 1 as well as from Tevatron by the CDF and D0 using proton-antiproton ($p\Bar{p}$) collisions at $\sqrt{s} = 1.96$ TeV.


% \include{event_selection_obj_reco}
% \include{background}
% \include{signal_clockwork}
% \include{systematics}
% \include{results}
% \newpage
% \section{References}
% \renewcommand{\bibsection}{}%removes the spaces and unwanted references heading from the list
% \begin{singlespacing}
% \bibliographystyle{apsrev}
% \bibliography{Ref_Introduction.bib}
% \end{singlespacing}\par

