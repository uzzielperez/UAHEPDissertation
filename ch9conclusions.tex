\chapter{\textnormal{Conclusions}}~\label{ch:conclusion}
This dissertation presented the CMS Run~2 legacy results for a search for non-resonant signature of Beyond the Standard Model (BSM) physics in high-mass diphoton events produced from pp collisions at a center-of-mass energy of 13 \TeV using CMS data corresponding to 138~\fbinv. We considered three model interpretations which are the Arkani-Hamed, Dimopoulos, and Dvali (ADD) Large Extra Dimensions, the Clockwork/Linear Dilaton and the Unparticles. These models propose solutions to the hierarchy problem and the Higgs Naturalness Problem through the existence of extra spatial dimensions or a scale invariant sector that closes the gap between the electroweak and the illusory Planck Scale.

The data are consistent with the predictions of the Standard model and exclusion limits are placed on the ADD and CW model at 95\% confidence level. For the ADD model the limits are further pushed to the range of $7.1 < \Ms < 11.1\TeV$ for the various model conventions. Limits are also set in the two-dimensional space of the Continuum Clockwork Model, with the fundamental scale $M_5$ level below 8.0~\TeV for $k$ values in the range between 0.2~\GeV and 2.0~\TeV. These results extend the current best lower limits on \Ms and $M_5$ by CMS in the diphoton channel, as presented in Ref.~\cite{cmsdiphoton2016}. Additionally, preliminary studies on the expected limits for the Unparticles are also presented using a cut-and-count approach and the CLs method which is a method used set conservative confidence intervals on the parameters of interest which are scaling dimension $d_U$ and the energy scale $\Lambda_{U}$. 

Extensions of this result for Run~3 are expected. First, a model-independent search could be performed using an autoencoder-based anomaly detection procedure to search for generic non-resonant deviations from the SM background. Additionally, a multivariate analysis (MVA) can be used to enhance the cut-based photon ID, as well as considering other kinematic variables. The CMS dijet Searches for Extra Dimensions incorporated angular information into the analysis as an additional signal discriminator~\cite{CMS:2018mgb}. For the fake prediction, a 30\% non-closure contributes to a large uncertainty in our knowledge of the fake background normalization. The systematic uncertainties could also be reduced by using reprocessed data sets which include improved Electromagnetic Calorimeter (ECAL) calibration with better energy resolution~\cite{Cavallari:2020ydz}. Finally, the ATLAS collaboration~\cite{ATLAS:2023hbp} also suggested that since the non-resonant enhancements are narrowly spaced resonances undetected by even CMS's excellent ECAL detector resolution, the continuous wavelet transform method (CWT) could allow us to examine the LHC spectra in the frequency domain and not just on the mass space alone. The outputs of the CWT in the form of scalograms can then be used as inputs to neural networks such as a binary classifier and an autoencoder to improve the limits on model-dependent and model-independent analyses. These are just some of the possibilities that may be realized by future analyzers.


% You could add the idea of incorporating angular information into the analysis as a further signal discriminator, and give appropriate citations, such as the CMS dijet LED searches, and the Landsberg paper (from the time you started looking into this)
% CMS:2018mgb

At the time of writing this dissertation, the analysis presented here, in combination with a related search for resonant excesses in the high-mass diphoton spectrum, is in the final stages of internal CMS collaboration review before being submitted for publication in a peer-reviewed journal.

% The results of this dissertation comes along with an analogous search for resonant excess in the high-mass diphoton channel (\texttt{EXO-22-024}) is in the last stages of review prior to journal publication.

% An improved ECAL calibration for the full Run 2 dataset (2016-2018) was performed
% during 2019 to achieve optimal performance. The CMS experiment has used these calibration
% sets for the full Run2 data reprocessing, which has been carried out during Long shutdown
% 2. This reprocessing constitutes the “ultra-legacy” data set, which will be used for analyses
% requiring optimal energy resolution and will be preserved for future analyses on Run2 data.